% II Äußere Maße

\begin{frame}
\frametitle{Äußere Maße}
Sei $X$ eine Menge. Eine Funktion $\mu: \script{P}(X) \to [0,\infty]$ mit $\mu(\emptyset)=0$ heißt \textbf{äußeres Maß} auf X, falls gilt:
      \begin{center}
        $A \subseteq \bigcup\limits_{i \in \mathbb{N}} A_i \implies \mu(A) \leq \sum\limits_{i \in \mathbb{N}} \mu(A_i)$
      \end{center}
      \medskip
\begin{enumerate}[label=(\roman*)]
        \item[]
        \item Die Begriffe $\sigma$-additiv, $\sigma$-subadditiv, $\sigma$-endlich, endlich, monoton sowie Nullmenge und $\mu$-fast überall werden wie für Maße definiert. (Man ersetze überall $\script{A}$ durch $\script{P}(X)$)
        \item Ein äußeres Maß ist monoton, $\sigma$-subadditiv und insbesondere endlich subadditiv\\
          (d.h. $A \subseteq \bigcup\limits_{i=1}^n A_i \implies \mu(A) \leq \sum\limits_{i = 1}^n \mu(A_i)$)
      \end{enumerate}
\end{frame}


\begin{frame}
\frametitle{messbare Menge}
Sei $\mu$ äußeres Maß auf $X$. Die Menge $A \subseteq X$ heißt \textbf{$\bm{\mu}$-messbar}, falls $\forall S \subseteq X$ gilt:
      \begin{center}
        $\mu(S) \geq \mu(S \cap A) + \mu(S \setminus A)$.
      \end{center}
      Das System aller $\mu$-messbaren Mengen wird mit $\bm{\script{M}(\mu)}$ bezeichnet.\\
      \medskip
Da $S = (S \cap A) \cup (S \setminus A)$ folgt aus Def. II.1:
      \begin{center}
        $\mu(S) \leq \mu(S \cap A) + \mu(S \setminus A)$
      \end{center}
      d.h.: $A$ messbar $\Leftrightarrow \mu(S \cap A) + \mu(S \setminus A) \ \forall S \subseteq X$
\end{frame}


\begin{frame}
\frametitle{$\mu$ als äußeres Maß}
Sei $\script{Q}$ ein System von Teilmengen einer Menge $X$, welches die leere Menge enthält, und sei $\lambda: \script{Q} \to [0,\infty]$ eine Mengenfunktion auf $\script{Q}$ mit $\lambda(\emptyset)=0$. Definiere die Mengenfunktion $\mu(E):= \inf\{\sum\limits_{i \in \mathbb{N}} \lambda(P_i) |\ P_i \in \script{Q}, E \subseteq \bigcup\limits_{i \in \mathbb{N}} P_i\}$.\\
      Dann ist $\mu$ ein äußeres Maß. \hfill ($\inf \emptyset = \infty$)
\end{frame}


\begin{frame}
\frametitle{Einschränkung}
Sei $\mu: \script{P}(X) \to [0, \infty]$ äußeres Maß auf X. Für M $\subseteq X$ gegeben erhält man durch $\mu \llcorner M: \script{P}(X) \to [0, \infty], \mu \llcorner M(A) := \mu(A \cap M)$ ein äußeres Maß $\mu \llcorner M$ auf $X$, welches wir \textbf{Einschränkung} von $\mu$ auf M nennen.\\
      Es gilt:
      \begin{center}
        $A$ $\mu$-messbar $\implies$ $A$ $\mu \llcorner M$-messbar
      \end{center} 
\end{frame}


\begin{frame}
\frametitle{Satz II.5}
$\mu$ äußeres Maß auf $X$. Dann gilt:
      \begin{align*}
        N \ \mu\text{-Nullmenge} &\implies N \ \mu\text{-messbar}\\
        N_k, k \in \mathbb{N}, \mu\text{-Nullmengen} &\implies \bigcup\limits_{k \in \mathbb{N}} N_k \ \mu\text{-Nullmenge}
      \end{align*}
$\script{M}(\mu)$ enthält alle Nullmengen $N \subseteq X$ und damit auch deren Komplemente\\
      (siehe Satz II.7). Es kann sein, dass keine anderen Mengen $\mu$-messbar sind.
\end{frame}

% Vorlesung 5

\begin{frame}
\frametitle{Lemma II.6}
Seien $A_i \in \script{M}(\mu)$, $i=1,...,k$, paarweiße disjunkt und $\mu$ äußeres Maß. Dann gilt $\forall S \subseteq X:$
      \begin{center}
        $\mu(S \cap \bigcup\limits_{i=1}^k A_i) = \sum\limits_{i=1}^k \mu(S \cap A_i)$
      \end{center}
\end{frame}


\begin{frame}
\frametitle{Satz II.7}
Sei $\mu: \script{P}(X) \to [0,\infty]$ ein äußeres Maß. Dann ist $\script{M}(\mu)$ eine $\sigma$-Algebra und $\mu$ ist ein vollständiges Maß auf $\script{M}(\mu)$.
\end{frame}


\begin{frame}
\frametitle{Lemma II.8}
$\mu$ äußeres Maß, $A_i \in \script{M}(\mu), i \in \mathbb{N}$.\\
      Dann gelten:
      \begin{enumerate}[label=\roman*)]
        \item Aus $A_1 \subseteq ... \subseteq A_i \subseteq A_{i+1} \subseteq ...$ folgt $\mu(\bigcup\limits_{i \in \mathbb{N}} A_i) = \lim\limits_{i \to \infty} \mu(A_i)$
        \item Aus $A_1 \supseteq ... \supseteq A_i \supseteq A_{i+1} \supseteq ...$ mit $\mu(A_1) < \infty$ folgt $\mu(\bigcap\limits_{i \in \mathbb{N}} A_i) = \lim\limits_{i \to \infty} \mu(A_i)$
      \end{enumerate} 
\end{frame}


\begin{frame}
\frametitle{Def. II.9 / X-stabil}
Ein Mengensystem $\script{A} \subseteq \script{P}(X)$ heißt $\bm{\bigcup}$\textbf{-stabil} (bzw. $\bm{\bigcap}$\textbf{-stabil}, $\bm{\setminus}$\textbf{-stabil}), wenn $A \cup B \in \script{A}$ (bzw. $A \cap B \in \script{A}$, $A \setminus B \in \script{A}$) $\forall A,B \in \script{A}$ gilt.\\
\medskip
$\bigcup$-stabil impliziert Stabilität bzgl. endlicher Vereinigung. Ebenso $\bigcap$-stabil.
\end{frame}


\begin{frame}
\frametitle{Def. II.10 / Ring / Algebra}
Ein Mengensystem $\script{R}\subset\script{P}(X)$ heißt \textbf{Ring} über $X$, falls:
      \begin{enumerate}[label=\roman*)]
        \item $\emptyset \in \script{R}$
        \item $A,B \in \script{R} \implies A \setminus B \in \script{R}$
        \item $A,B \in \script{R} \implies A \cup B \in \script{R}$
      \end{enumerate}
      
      $\script{R}$ heißt \textbf{Algebra}, falls zusätzlich $X \in \script{R}$.\\
      \medskip
Für $A,B \in \script{R}$ gilt: $A \cap B = A \setminus (A \setminus B) \in \script{R}$\\
      Ringe sind $\bigcup$-stabil, $\bigcap$-stabil, $\setminus$-stabil
\end{frame}


\begin{frame}
\frametitle{Def. II.11 (Im Aufschrieb II.10) / Prämaß}
Sei $\script{R} \subseteq \script{P}(X)$ Ring. Eine Funktion $\lambda: \script{R} \to [0, \infty]$ heißt \textbf{Prämaß} auf $\script{R}$, falls:
      \begin{enumerate}[label=\roman*)]
        \item $\lambda(\emptyset) = 0$
        \item Für $A_i \in \script{R}, i \in \mathbb{N}$, paarweiße disjunkt mit $\bigcup\limits_{i \in \mathbb{N}} A_i \in \script{R}$ gilt:\\
        $\lambda(\bigcup\limits_{i \in \mathbb{N}} A_i) = \sum\limits_{i \in \mathbb{N}} \lambda(A_i) $
      \end{enumerate}
      \medskip
$\sigma$-subadditiv, subadditiv, $\sigma$-endlich, endlich, monoton, Nullmenge und fast-überall werden wie für Maße definiert.
\end{frame}


\begin{frame}
\frametitle{Def. II.12 (Im Aufschrieb II.11) / Fortsetzung}
$\lambda$ Prämaß auf Ring $\script{R} \subseteq \script{P}(X)$. Ein äußeres Maß $\mu$ auf $X$ (bzw. ein Maß auf $\script{A}$) heißt \textbf{Fortsetzung} von $\lambda$, falls gilt:
      \begin{enumerate}[label=\roman*)]
        \item $\mu|_\script{R} = \lambda$, d.h. $\mu(A) = \lambda(A) \ \forall A \in \script{R}$
        \item $\script{R} \subseteq \script{M}(\mu)$ (bzw. $\script{R} \subset \script{A}$), d.h. alle $A \in \script{R}$ sind $\mu$-messbar
      \end{enumerate}
\end{frame}


\begin{frame}
\frametitle{induziertes äußeres Maß / Caratheodory-Fortsetzung}

\end{frame}


\begin{frame}
\frametitle{Lemma II.14 (Im Aufschrieb II.13)}
$\lambda: \script{R} \to [0, \infty]$ Prämaß auf Ring $\script{R} \subseteq \script{P}(X)$. Sei $\mu: \script{P}(X) \to [0, \infty]$ das in Satz II.3 aus $\script{R}$ konstruierte äußere Maß, d.h. $\forall E \subseteq X:$
      \begin{align*}
        \mu(E) := inf\{\sum\limits_{i \in \mathbb{N}} \lambda(A_i) \ | \ A_i \in \script{R}, E \subseteq \bigcup\limits_{i \in \mathbb{N}} A_i\}
      \end{align*}
      Dann ist $\mu$ eine Fortsetzung von $\lambda$.\\
      $\mu$ heißt \textbf{induziertes äußeres Maß} oder \textbf{Caratheodory-Fortsetzung} von $\lambda$.
\end{frame}

% Vorlesung 6

\begin{frame}
\frametitle{}
Sei $\lambda: \script{R} \to [0, \infty]$ Prämaß auf Ring $\script{R}\subseteq \script{P}(X)$. Dann ex. ein Maß $\mu$ auf $\sigma(\script{R})$ mit $\mu=\lambda$ auf $\script{R}$. Diese Fortsetzung ist eindeutig, falls $\lambda$ $\sigma$-endlich ist.
\end{frame}


\begin{frame}
\frametitle{Regularität der Caratheodory-Fortsetzung / i.A. II.15}
Sei $\mu$ Caratheodory-Fortsetzung des Prämaßes $\lambda: \script{R} \to [0,\infty]$ auf Ring $\script{R}$ über $X$. Dann ex. $\forall D \subseteq X$ ein $E \in \sigma(\script{R})$ mit $E \supseteq D$ und $\mu(E) = \mu(D)$.\\
    ($\mu$ ist \glqq reguläres \grqq äußeres Maß)
\end{frame}

\begin{frame}
\frametitle{Satz II.17 (i.A. II.16)}
Sei $\lambda$ ein $\sigma$-endliches Prämaß auf Ring $\script{R}$ über $X$ und sei $\mu: \script{P}(X) \to [0,\infty]$ die Caratheodory-Fortsetzung von $\lambda$. Dann ist $\mu|_{\script{M}(\mu)}$ die Vervollständigung von $\mu|_{\sigma(\script{R})}$ und $\script{M}(\mu)$ ist die vervollständigte $\sigma$-Algebra von $\overline{\sigma(\mathbb{R})}_{\mu|_{\sigma(\mathbb{R})}}$.\\
    D.h. $\overline{\sigma(\mathbb{R})}_{\mu|_{\sigma(\mathbb{R})}} = \script{M}(\mu)$. Insbesondere ex. genau eine Fortsetzung von $\lambda: \script{R} \to [0, \infty]$ zu einem vollständigen Maß auf $\script{M}(\mu)$.
\end{frame}


\begin{frame}
\frametitle{Lemma II.18 (i.A. II.17)}
$\lambda: \script{R} \to [0, \infty]$ $\sigma$-endliches Prämaß auf Ring $\script{R} \subseteq \script{P}(X)$ mit Caratheodory-Fortsetzung $\mu$. $D \subseteq X$ ist genau dann $\mu$-messbar, wenn eine der folgenden Bedingungen gilt:
    \begin{enumerate}[label=\roman*)]
      \item $\exists \ E \in \sigma(\script{R})$ mit $E \supseteq D$ und $\mu(E \setminus D) = 0$
      \item $\exists \ C \in \sigma(\script{R})$ mit $C \subseteq D$ und $\mu(D \setminus C) = 0$
    \end{enumerate}
\end{frame}

\begin{frame}
\frametitle{Def. II.19 / Halbring}
Ein Mengensystem $\script{Q} \subseteq \script{P}(X)$ heißt \textbf{Halbring} über $X$, falls:
    \begin{enumerate}[label=\roman*)]
      \item $\emptyset \in \script{Q}$
      \item $P, Q \in \script{Q} \implies P \cap Q \in \script{Q}$
      \item $P, Q \in \script{Q} \implies P \setminus Q = \bigcup\limits_{i=1}^k P_i$ mit endlich vielen paarweise disjunkten $P_i \in \script{Q}$
    \end{enumerate}
\end{frame}


\begin{frame}
\frametitle{Bemerkung: Intervall / Quader}

\end{frame}

\begin{frame}
\frametitle{Satz II.20 (i.A. II.19)}
$\script{I}$ ist ein Halbring.
\end{frame}


\begin{frame}
\frametitle{Satz II.21 (i.A. II.20)}
Für $i = 1, ..., n$ sei $\script{Q}_i$ Halbring über $X_i$. Dann ist $\script{Q}:=\{P_1 \times ... \times P_n \ | \ P_i \in \script{Q}_i\}$ ein Halbring über $X_1 \times ... \times X_n$.
\end{frame}

\begin{frame}
\frametitle{Satz II.22 (i.A. II.21)}
$\script{Q}^n$ ist ein Halbring.
\end{frame}

% Vorlesung 7

\begin{frame}
\frametitle{Satz II.23 (i.A. II.22)}
$\script{Q}$ Halbring über $X$ und $\script{F}$ sei das System aller endlichen Vereinigungen $F=\bigcup\limits_{i=1}^k P_i$ von Mengen $P_I \in \script{Q}$. Dann ist $\script{F}$ der von $\script{Q}$ erzeugte Ring.
\end{frame}

\begin{frame}
\frametitle{Figuren}
$\script{Q} := \{\emptyset\} \cup \{\{a\} \ | \ a \in X\}$\\
            $\implies$ erzeugter Ring $\script{F}$: Ring der endlichen Teilmengen von $X$.
\end{frame}


\begin{frame}
\frametitle{Lemma II.24 (i.A. II.23)}
$\script{Q}$ Halbring über $X$, $\script{F}$ der von $\script{Q}$ erzeugte Ring. $\implies \sigma(\script{Q}) = \sigma(\script{F})$
\end{frame}


\begin{frame}
\frametitle{Lemma II.25 (i.A. II.24)}
$\script{Q}$ Halbring über $X$, $\script{F}$ der von $\script{Q}$ erzeugte Ring. Zu jedem $F \in \script{F}$ existieren paarweise disjunkte $P_1, ..., P_k \in \script{Q}$ mit $F = \bigcup\limits_{i=1}^k P_i$
\end{frame}


\begin{frame}
\frametitle{Def. II.26 (i.A. II.25) / Inhalt}
Sei $\script{Q} \subseteq \script{P}(X)$ Halbring. Eine Funktion $\lambda: \script{Q} \to [0, \infty]$ heißt \textbf{Inhalt} auf $\script{Q}$, falls:
    \begin{enumerate}[label=\roman*)]
      \item $\lambda(\emptyset) = 0$
      \item Für $A_i \in \script{Q}$ paarweiße disjunkt mit $\bigcup\limits_{i=1}^n A_i \in \script{Q}$ gilt: $\lambda(\bigcup\limits_{i=1}^n A_i) = \sum\limits_{i=1}^n \lambda(A_i)$
    \end{enumerate}
    $\lambda$ heißt \textbf{Prämaß} auf $\script{Q}$, falls $\lambda$ $\sigma$-additiv auf $\script{Q}$ ist.\\
    D.h. für $A_i \in \script{Q}$ paarweiße disjunkt ($i \in \mathbb{N}$) mit $\bigcup\limits_{i \in \mathbb{N}} A_i \in \script{Q}: \lambda(\bigcup\limits_{i \in \mathbb{N}} A_i) = \sum\limits_{i \in \mathbb{N}} \lambda(A_i)$
\end{frame}


\begin{frame}
\frametitle{Satz II.27 (i.A. II.26)}
$\lambda$ Inhalt auf Halbring $\script{Q}$ und $\script{F}$ der von $\script{Q}$ erzeugte Ring. Dann ex. genau ein Inhalt $\bar{\lambda}:\script{F} \to [0, \infty]$ mit $\bar{\lambda}(Q)=\lambda(Q) \ \forall Q \in \script{Q}$.
\end{frame}


\begin{frame}
\frametitle{Lemma II.28 (i.A. II.27)}
$\lambda$ Inhalt auf Halbring $\script{Q}$ über $X$\\
    $\implies \lambda$ ist monoton und subadditiv
\end{frame}


\begin{frame}
\frametitle{Satz II.29 (i.A. II.28)}
$vol^n(.)$ ist ein Inhalt auf $\script{Q}^n$
\end{frame}


\begin{frame}
\frametitle{Satz II.30 (i.A. II.29)}
$\lambda: \script{Q} \to [0, \infty]$ Prämaß auf Halbring $\script{Q} \subseteq \script{P}(X)$, $\script{R}$ der von $\script{Q}$ erzeugte Ring und $\bar{\lambda}: \script{R} \to [0,\infty]$ der eindeutig bestimmte Inhalt auf $\script{R}$ mit $\bar{\lambda}|_{\script{Q}}=\lambda$ (Satz II.27 / i.A. II.26), so ist $\bar{\lambda}$ ein Prämaß auf $\script{R}$.
\end{frame}

% Vorlesung 8

\begin{frame}
\frametitle{Satz II.31 ((i.A. II.30))}
$\lambda: \script{Q} \to [0, \infty]$ Prämaß auf Halbring $\script{Q} \subseteq \script{P}(X)$. Sei $\mu:\script{P}(X) \to [0,\infty]$ das in Satz II.3 aus $\script{Q}$ konstruierte äußere Maß, d.h. $\forall E \subseteq X$ ist:
    \begin{align*}
      \mu(E) = inf\{\sum\limits_{i \in \mathbb{N}} \lambda(A_i) \ | \ A_i \in \script{Q}, E \subseteq \bigcup\limits_{i \in \mathbb{N}}A_i \}
    \end{align*}
    Dann ist $\mu$ eine Fortsetzung von $\lambda$.
\end{frame}


\begin{frame}
\frametitle{Satz II.32 ((i.A. II.31))}
Für einen Inhalt $\lambda$ auf Ring $\script{R}$ und $A_i \in \script{R}, i \in \mathbb{N}$, betrachte:
    \begin{enumerate}[label=\roman*)]
      \item $\lambda$ ist Prämaß auf $\script{R}$
      \item Für $A_i \subseteq A_{i+1} \subseteq ...$ mit $\bigcup\limits_{i \in \mathbb{N}} A_i \in \script{R}$ gilt: $\lambda(\bigcup\limits_{i \in \mathbb{N}} A_i) = \lim\limits_{n \to \infty} \lambda(A_n)$
      \item Für $A_i \supseteq A_{i+1} \supseteq ...$ mit $\lambda(A_1) < \infty$ und $\bigcap\limits_{i \in \mathbb{N}} A_i \in \script{R}$ gilt:\\
      $\lambda(\bigcap\limits_{i \in \mathbb{N}} A_i) = \lim\limits_{n \to \infty} \lambda(A_n)$
      \item Für $A_i \supseteq A_{i+1} \supseteq ...$ mit $\lambda(A_1) < \infty$ und $\bigcap\limits_{i \in \mathbb{N}} A_i = \emptyset$ gilt: $\lim\limits_{i \to \infty} \lambda(A_i) = 0$
    \end{enumerate} 
    Dann gilt: i) $\Leftrightarrow$ ii) $\implies$ iii) $\implies$ iv)\\
    Ist $\lambda$ endlich, d.h. $\lambda(A) < \infty \ \forall A \in \script{R}$, dann sind i) - iv) äquivalent.
\end{frame}



