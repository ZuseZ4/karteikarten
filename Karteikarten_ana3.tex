\documentclass[fleqn]{beamer}
\usepackage[ngerman]{babel}
\usepackage[utf8]{inputenc}
\usetheme{default}
\usepackage{amssymb}
\usepackage{amsthm}
\usepackage{amsmath}
\usepackage{mdframed}
\usepackage{trfsigns}
\usepackage{cancel}
\usepackage{graphicx}
\usepackage{float}
\usepackage{enumitem}
\usepackage{lmodern}
%\setbeamertemplate{enumerate item}{(\roman{enumi})}
%\setbeamertemplate{enumerate item}{(\alph{enumi})}

\usepackage{bm}
\begin{document}
\newcommand{\script}[1]{\mathcal{#1}}

% Maße und messbare Funktionen
\begin{frame}
\frametitle{Def I.1, $\sigma$-Algebra, messbarer Raum}
Menge $X$, Potenzmenge $\script{P}(X)$, eine Teilmenge von 
$\script{P}(X)$ heißt Mengensystem

Ein Mengensystem $\script{A} \subseteq \script{P}(X)$ heißt
\textbf{$\bm\sigma$-Algebra}, falls:
\begin{enumerate}[label=(\roman*)]
    \item $X \in \script{A}$
    \item $A \in \script{A} \implies X \setminus A \in \script{A}$
    \item $A_i \in \script{A}, \forall i \in \mathbb{N} \implies
    \bigcup\limits_{i \in \mathbb{N}} A_i \in \script{A}$
\end{enumerate}
Das Paar $(X, \script{A})$ heißt dann \textbf{messbarer Raum}.\\
\end{frame}

\begin{frame}
\frametitle{Satz I.2}
Jeder Durchschnitt von (endlich oder unendlich vielen) $\sigma$-Algebren auf der selben Menge X ist wieder eine $\sigma$-Algebra.
\end{frame}

\begin{frame}
\frametitle{Def. I.3}
      Für ein Mengensystem $\script{E} \subseteq \script{P}(X)$ heißt $\sigma(\script{E}) := \bigcap\{\script{A}|\script{A}$ ist $\sigma$-Algebra in $X$ mit $\script{E} \subseteq \script{A}\}$ die von $\script{E}$ \textbf{erzeugte $\bm{\sigma}$-Algebra}. Man nennt $\script{E}$ das \textbf{erzeugende System} von $\sigma(\script{E})$.\\
      \medskip
            Dieser Durchschnitt ist nicht-trivial, denn $\script{P}(X)$ ist $\sigma$-Algebra mit $\script{E} \subseteq \script{P}(X)$.
\end{frame}

\begin{frame}
\frametitle{Def. I.4}
   Eine Folge $(s_k) \subseteq \bar{\mathbb{R}}\ (k \in \mathbb{N})$ konvergiert gegen $s \in \bar{\mathbb{R}}$, falls eine der folgenden Alternativen gilt:
      \begin{enumerate}[label=(\roman*)]
        \item $s \in \mathbb{R}$ und $\forall \epsilon>0$ gilt: $s_k \in (s-\epsilon, s+\epsilon) \subseteq \mathbb{R}$ für k hinreichend groß
        \item $s=\infty$ und $\forall r \in \mathbb{R}: s_k \in (r, \infty]$ für k hinreichend groß
        \item $s=-\infty$ und $\forall r \in \mathbb{R}: s_k \in [-\infty, r)$ für k hinreichend groß
      \end{enumerate}
      $(s_k) \subseteq \mathbb{R}$ ist genau dann in $\bar{\mathbb{R}}$ konvergent, wenn sie entweder in $\mathbb{R}$ konvergiert, oder bestimmt gegen $\pm\infty$ divergiert.
\end{frame}


\begin{frame}
\frametitle{Def. I.5, Maßraum}
Sei $\script{A} \subseteq \script{P}(X)$ eine $\sigma$-Algebra, eine nicht-negative Mengenfunktion $\mu: \script{A} \rightarrow [0, \infty]$ heißt \textbf{Maß} auf $\script{A}$, falls:
      \begin{enumerate}[label=(\roman*)]
        \item $\mu(\emptyset) = 0$
        \item für beliebige paarweiße disjunkte $A_i \in \script{A}$, $i \in \mathbb{N}$, gilt:\\
              $\mu(\bigcup\limits_{i \in \mathbb{N}} A_i) = \sum\limits_{i \in \mathbb{N}} \mu (A_i)$ \hfill ($\sigma$-Additivität)
      \end{enumerate}
      Das Tripel $(X, \script{A}, \mu)$ heißt \textbf{Maßraum}.\\
      \medskip
      Bem.:
      \begin{enumerate}[label=(\roman*)]
        \item Für endlich viele paarweiße disjunkte $A_i \in \script{A}$, $i=1,...,n$, folgt aus (ii) indem man $A_i=\emptyset$ für $i=n+1, ...$ setzt: $\mu(\bigcup\limits_{i=1}^n A_i) = \sum\limits_{i=1}^n \mu(A_i)$
        \item Monotonie des Maßes: $A,B \in \script{A}$ mit $A \subseteq B \implies \mu(A) \leq \mu(B) = \mu(A \cup (B \setminus A)) = \mu(A) + \mu(B \setminus A)$
      \end{enumerate}
\end{frame}


\begin{frame}
\frametitle{Def. I.6}
      Sei $(X, \script{A}, \mu)$ ein Maßraum. Das Maß $\mu$ heißt \textbf{endlich}, wenn $\mu(A) < \infty\ \forall A \in \script{A}$ und $\bm{\sigma}$\textbf{-endlich}, wenn es eine Folge $(X_i) \in \script{A}$ mit $\mu(X_i) < \infty$ gibt, sodass $X=\bigcup\limits_{i \in \mathbb{N}} X_i$. Falls $\mu(X) = 1$, so wird $\mu$ \textbf{Wahrscheinlichkeits-Maß} genannt.
\end{frame}


\begin{frame}
\frametitle{Satz I.7 (Stetigkeitseig. von Maßen)}
Sei $(X,\script{A},\mu)$ Maßraum. Dann gelten für Mengen $A_i \in \script{A}, i \in \mathbb{N}$ folgende Aussagen:

      \begin{enumerate}[label=(\roman*)]
        \item Aus $A_1 \subseteq A_2 \subseteq A_3 \subseteq ...$ folgt: $\mu (\bigcup\limits_{i \in \mathbb{N}} A_i) = \lim\limits_{i \to \infty} \mu (A_i)$
        \item Aus $A_1 \supseteq A_2 \supseteq A_3 \supseteq ...$ mit $\mu(A_1)<\infty$, folgt: $\mu (\bigcap\limits_{i \in \mathbb{N}} A_i) = \lim\limits_{i\to \infty} \mu (A_i)$
        \item $\mu(\bigcup\limits_{i\in \mathbb{N}} A_i) \leq \sum\limits_{i\in \mathbb{N}} \mu(A_i)$
      \end{enumerate}
\end{frame}


\begin{frame}
\frametitle{Bemerkungen zu Satz I.7}
\begin{enumerate}[label=(\arabic*)]
        \item \begin{enumerate}[label=(\roman*)]
          \item Stetigkeit von unten
          \item Stetigkeit von oben
          \item $\sigma$-Subadditivität von $\mu$
        \end{enumerate}
        \item Bedingung $\mu(A_i) \leq \infty$ in (ii) kann durch $\mu(A_k) \leq \infty$ für ein $k \in \mathbb{N}$ ersetzt werden, kann aber nicht weggelassen werden.\\
              Begründung:\\
              $A_k = {k, k+1,...} \subseteq \mathbb{N}$\\
              $card(A_k) = \infty \ \forall k \in \mathbb{N}$\\
              Aber: $card(\bigcap\limits_{i \in \mathbb{N}} A_i) = card(\emptyset) = 0$ 
      \end{enumerate}
\end{frame}


\begin{frame}
\frametitle{Def. I.8}
$(X, \script{A}, \mu)$ Maßraum.\\
      Jede Menge $A \in \script{A}$ mit $\mu(A) = 0$ heißt $\bm{\mu}$\textbf{-Nullmenge}. Das System aller $\mu$-Nullmengen bezeichnen wir mit $\bm{\script{N}(\mu)}$. Das Maß $\mu$ heißt \textbf{vollständig}, wenn gilt:\\
      \[
        N \subseteq A \text{ für ein } H \in \script{A} 
        \text{ mit } \mu(A)=0 
        \implies N \in \script{A} \text{ und } \mu(N)=0
      \]
      \medskip
      Bem.: 
      Nicht jedes Maß ist vollständig:
      \[
        \script{A} \neq \script{P}(X) \ \mu(A) = 0 \ \forall A \in \script{A}
      \]
      Allerdings lässt sich jedes Maß vervollständigen
\end{frame}


\begin{frame}
\frametitle{Zu Def. I.8: Vervollstandigung}
$\bar{\mu}$ ist wohldefiniert: $A \cup N = B \cup P$ mit $A,B\in \script{A}, \ P,N \in \script{T}_{\mu} \implies \exists C \in \script{A}, \mu(C) = 0: P \subseteq C \implies A \subseteq B \cup C \implies \mu(A) \leq \mu(B) + \mu(C) = \mu(B)$\\
      Symm $\implies \mu(A) = \mu(B)$\\
      $\bar{\mu}$ heißt \textbf{Vervollständigung} von $\mu$
\end{frame}


\begin{frame}
\frametitle{Satz I.9}
$(X,\script{A}, \mu)$ Maßraum. Dann ist $\bar{\script{A}}_{\mu}$ eine $\sigma$-Algebra und $\bar{\mu}$ ein vollständiges Maß auf $\bar{\script{A}}_{\mu}$, welches mit $\mu$ auf $\script{A}$ übereinstimmt.
\end{frame}


\begin{frame}
\frametitle{Satz I.10}
$(X, \script{A}, \mu)$ Maßraum und $(X, \bar{\script{A}}_{\mu}, \bar{\mu})$ sei Vervollständigung. Ferner sei $(X, \script{B}, \nu)$ ein vollständiger Maßraum mit $\script{A} \subseteq \script{B}$ und $\mu = \nu$ auf $\script{A}$. Dann ist $\bar{\script{A}}_{\mu} \subseteq \script{B}$ und $\bar{\mu} = \nu$ auf $\bar{\script{A}}_{\mu}$.
\end{frame}


\begin{frame}
\frametitle{Def. I.11}
$(X, \script{A}), (Y, \script{C})$ messbare Räume. Eine Abbildung $f: X \to Y$ heißt $\bm{\script{A}-\script{C}-}$\textbf{messbar}, falls $f^{-1}(\script{C}) \subseteq{\script{A}}$\\
Falls $\script{A}, \script{C}$ klar sind, bezeichnen wir $f$ einfach als messbar

\end{frame}


\begin{frame}
\frametitle{Lemma I.12}
$(X, \script{A}), (Y, \script{C})$ messbare Räume und $\script{C} := \sigma(\script{E})$. Jede Abbildung $f: X \to Y$ mit $f^{-1}(\script{E}) \subseteq \script{A}$ ist $\script{A}$-$\script{C}$-messbar
\end{frame}



\begin{frame}
\frametitle{borel-messbar (Zu Lemma I.12)}
Jede stetige Abbildung $f: \mathbb{R}^n \to \mathbb{R}^n$ ist $\mathbb{B}^n$-$\mathbb{B}^n$-messbar\\
              (man sagt: $f$ ist \textbf{borel-messbar}).\\
              Denn $\mathbb{B}^n = \sigma(\{\text{offene Teilmengen des } \mathbb{R}^n\})$ und Urbilder offener Mengen sind offen für $f$ stetig (siehe. Ana 1)
\end{frame}


\begin{frame}
\frametitle{Def. I.13}
$(X, \script{A})$ messbarer Raum und $D \in \script{A}$.\\
      Eine Funktion $f: D \to \bar{\mathbb{R}}$ heißt \textbf{numerische Funktion}.
\end{frame}


\begin{frame}
\frametitle{Lemma I.14}
$(X, \script{A})$ messbarer Raum, $D \in \script{A}$ und $f: D \to \bar{\mathbb{R}}$.\\
      Dann sind folgende Aussagen äquivalent:
      \begin{enumerate}[label=(\roman*)]
        \item $f$ ist $\script{A}$-$\bar{\mathbb{B}}^1$-messbar
        \item $\forall \ \script{U} \subseteq \mathbb{R}$ offen ist $f^{-1}(\script{U}) \in \script{A}$ und $f^{-1}(\{\infty\}), f^{-1}(\{-\infty\}) \in \script{A}$
        \item $\{f \leq s\} := \{x \in D \ |\ f(x) \in [-\infty, s]\} \in \script{A} \ \forall s \in \mathbb{R}$
        \item $\{f < s\} := \{x \in D \ |\ f(x) \in [-\infty, s)\} \in \script{A} \ \forall s \in \mathbb{R}$
        \item $\{f \geq s\} := \{x \in D \ |\ f(x) \in [s, \infty]\} \in \script{A} \ \forall s \in \mathbb{R}$
        \item $\{f > s\} := \{x \in D \ |\ f(x) \in (s, \infty]\} \in \script{A} \ \forall s \in \mathbb{R}$
      \end{enumerate}
      \medskip
      In (iii) - (vi) reicht es aus, $s \in \mathbb{Q}$, statt $s \in \mathbb{R}$ zu haben, denn es gilt z.B.:\\
      $\{f \geq s\} = \bigcap\limits_{\stackrel{q \in \mathbb{Q}}{s > q}} \{f > q\}$
\end{frame}

% VL 3:

\begin{frame}
\frametitle{Lemma I.15}
Sei $(X, \script{A})$ ein messbarer Raum, $D \in \script{A}$ und $f,g: D \to \bar{\mathbb{R}}$ $\script{A}$-messbar. Dann sind die Mengen $\{f < g\} := \{x \in D: f(x) < g(x)\}$ und $\{f \leq g\} := \{x \in D: f(x) \leq g(x)\}$ Elemente aus $\script{A}$.
\end{frame}


\begin{frame}
\frametitle{Satz I.16}
$(X, \script{A})$ messbarer Raum, $D \in \script{A}$ und $f_k:D \to \bar{\mathbb{R}}$ Folge von $\script{A}$-messbaren Funktionen.\\
      Dann sind auch folgende Funktionen $\script{A}$-messbar:
      \begin{center}
        $\inf\limits_{k \in \mathbb{N}} f_k, \ \sup\limits_{k \in \mathbb{N}} f_k, \ \liminf\limits_{k \to \infty} f_k, \ \limsup\limits_{k \to \infty} f_k$
      \end{center}
\end{frame}


\begin{frame}
\frametitle{Satz I.17}
$(X, \script{A})$ messbarer Raum, $D \in \script{A}$, $f,g: D \to \bar{\mathbb{R}}$ $\script{A}$-messbar, $\alpha \in \mathbb{R}$.\\
      Dann sind die Funktionen
      \begin{center}
        $f+g, \ \alpha f, \ f^{\pm}, \ max(f,g), \ min(f,g), \ |f|, \ fg, \ \dfrac{f}{g}$
      \end{center}
      auf ihren Definitionsbereichen, die in $\script{A}$ liegen $\script{A}$-messbar.
\end{frame}


\begin{frame}
\frametitle{Def I.18}
$(X, \script{A}, \mu)$ Maßraum. Eine auf $D \in \script{A}$ definierte Funktion $f: D \to \bar{\mathbb{R}}$ heißt\\
      $\bm{\mu}$\textbf{-messbar} (auf $X$), wenn $\mu(X \setminus D) = 0$ und $f$ $\script{A|_D}$-messbar ist.\\
      ($\script{A}|_D := \{A \cap D | A \in \script{A}\}$, siehe Blatt 1)
\end{frame}


\begin{frame}
\frametitle{$\mu$-fast überall}
Sei $(X, \script{A}, \mu)$ Maßraum. Man sagt, die Aussage $A[x]$
ist wahr \textbf{für $\bm{\mu}$-fast alle} $x \in M \in \script{A}$
oder \textbf{$\bm{\mu}$-fast überall} auf M, 
falls es eine $\mu$-Nullmenge $N$ gibt mit
\begin{center}
$\{x \in M: A[x] \text{ ist falsch}\} \subseteq N$
\end{center}
Dabei wird nicht verlangt, dass $\{x \in M: A[x] 
\text{ ist falsch}\}$ selbst zu $\script{A}$ gehört.\\
Zum Beispiel bedeutet für Funktionen $f,g: X \to \bar{\mathbb{R}}$ 
die Aussage \glqq$f(x) \leq g(x)$ für $\mu$-fast alle $x \in X$
\grqq, dass es eine Nullmenge $N$ gibt, so dass $\forall x \in X 
\setminus N$ gilt: $f(x) \leq g(x)$.\\
Eine Funktion $h$ ist \glqq$\mu$-fast überall auf $X$ 
definiert\grqq, wenn $h$ auf $D \in \script{A}$ definiert ist 
und $\mu(X \setminus D) = 0$.\\
Ziel: 
Messbarkeit für Funktionen, die nur $\mu$-fast überall definiert sind.
\end{frame}

% Vorlesung 4

\begin{frame}
\frametitle{Lemma I.19}
$(X, \script{A}, \mu)$ vollständiger Maßraum. $f$ $\mu$-messbar auf $X$. Dann ist auch jede Funktion $\tilde{f}$ mit $\tilde{f}=f$ $\mu$-fast überall $\mu$-messbar.
\end{frame}


\begin{frame}
\frametitle{Satz I.20}
$(X, \script{A}, \mu)$ vollständiger Maßraum und seien $f_k, k \in \mathbb{N}$, $\mu$-messbar. Falls $f_k$ punktweise $\mu$-fast überall gegen $f$ konvergiert, dann ist $f$ auch $\mu$-messbar.
\end{frame}


\begin{frame}
\frametitle{Satz I.21 (Egorov)}
$(X, \script{A}, \mu)$ Maßraum, $D \in \script{A}$ Menge mit $\mu(D) < \infty$ und $f_n, f$ $\mu$-messbare, $\mu$-fast überall endliche Funktionen auf $D$ mit $f_n \to f$ $\mu$-fast überall. Dann existiert $\forall \epsilon > 0$ eine Menge $B \in \script{A}$ mit $B \subseteq D$ und
      \begin{enumerate}[label=(\roman*)]
        \item $\mu(D \setminus B) < \epsilon$
        \item $f_n \to f$ gleichmäßig auf $B$
      \end{enumerate}
\end{frame}

% II Äußere Maße

\begin{frame}
\frametitle{Äußere Maße}
Sei $X$ eine Menge. Eine Funktion $\mu: \script{P}(X) \to [0,\infty]$ mit $\mu(\emptyset)=0$ heißt \textbf{äußeres Maß} auf X, falls gilt:
      \begin{center}
        $A \subseteq \bigcup\limits_{i \in \mathbb{N}} A_i \implies \mu(A) \leq \sum\limits_{i \in \mathbb{N}} \mu(A_i)$
      \end{center}
      \medskip
\begin{enumerate}[label=(\roman*)]
        \item[]
        \item Die Begriffe $\sigma$-additiv, $\sigma$-subadditiv, $\sigma$-endlich, endlich, monoton sowie Nullmenge und $\mu$-fast überall werden wie für Maße definiert. (Man ersetze überall $\script{A}$ durch $\script{P}(X)$)
        \item Ein äußeres Maß ist monoton, $\sigma$-subadditiv und insbesondere endlich subadditiv\\
          (d.h. $A \subseteq \bigcup\limits_{i=1}^n A_i \implies \mu(A) \leq \sum\limits_{i = 1}^n \mu(A_i)$)
      \end{enumerate}
\end{frame}


\begin{frame}
\frametitle{messbare Menge}
Sei $\mu$ äußeres Maß auf $X$. Die Menge $A \subseteq X$ heißt \textbf{$\bm{\mu}$-messbar}, falls $\forall S \subseteq X$ gilt:
      \begin{center}
        $\mu(S) \geq \mu(S \cap A) + \mu(S \setminus A)$.
      \end{center}
      Das System aller $\mu$-messbaren Mengen wird mit $\bm{\script{M}(\mu)}$ bezeichnet.\\
      \medskip
Da $S = (S \cap A) \cup (S \setminus A)$ folgt aus Def. II.1:
      \begin{center}
        $\mu(S) \leq \mu(S \cap A) + \mu(S \setminus A)$
      \end{center}
      d.h.: $A$ messbar $\Leftrightarrow \mu(S \cap A) + \mu(S \setminus A) \ \forall S \subseteq X$
\end{frame}


\begin{frame}
\frametitle{$\mu$ als äußeres Maß}
Sei $\script{Q}$ ein System von Teilmengen einer Menge $X$, welches die leere Menge enthält, und sei $\lambda: \script{Q} \to [0,\infty]$ eine Mengenfunktion auf $\script{Q}$ mit $\lambda(\emptyset)=0$. Definiere die Mengenfunktion $\mu(E):= \inf\{\sum\limits_{i \in \mathbb{N}} \lambda(P_i) |\ P_i \in \script{Q}, E \subseteq \bigcup\limits_{i \in \mathbb{N}} P_i\}$.\\
      Dann ist $\mu$ ein äußeres Maß. \hfill ($\inf \emptyset = \infty$)
\end{frame}


\begin{frame}
\frametitle{Einschränkung}
Sei $\mu: \script{P}(X) \to [0, \infty]$ äußeres Maß auf X. Für M $\subseteq X$ gegeben erhält man durch $\mu \llcorner M: \script{P}(X) \to [0, \infty], \mu \llcorner M(A) := \mu(A \cap M)$ ein äußeres Maß $\mu \llcorner M$ auf $X$, welches wir \textbf{Einschränkung} von $\mu$ auf M nennen.\\
      Es gilt:
      \begin{center}
        $A$ $\mu$-messbar $\implies$ $A$ $\mu \llcorner M$-messbar
      \end{center} 
\end{frame}


\begin{frame}
\frametitle{Satz II.5}
$\mu$ äußeres Maß auf $X$. Dann gilt:
      \begin{align*}
        N \ \mu\text{-Nullmenge} &\implies N \ \mu\text{-messbar}\\
        N_k, k \in \mathbb{N}, \mu\text{-Nullmengen} &\implies \bigcup\limits_{k \in \mathbb{N}} N_k \ \mu\text{-Nullmenge}
      \end{align*}
$\script{M}(\mu)$ enthält alle Nullmengen $N \subseteq X$ und damit auch deren Komplemente\\
      (siehe Satz II.7). Es kann sein, dass keine anderen Mengen $\mu$-messbar sind.
\end{frame}

% Vorlesung 5

\begin{frame}
\frametitle{Lemma II.6}
Seien $A_i \in \script{M}(\mu)$, $i=1,...,k$, paarweiße disjunkt und $\mu$ äußeres Maß. Dann gilt $\forall S \subseteq X:$
      \begin{center}
        $\mu(S \cap \bigcup\limits_{i=1}^k A_i) = \sum\limits_{i=1}^k \mu(S \cap A_i)$
      \end{center}
\end{frame}


\begin{frame}
\frametitle{Satz II.7}
Sei $\mu: \script{P}(X) \to [0,\infty]$ ein äußeres Maß. Dann ist $\script{M}(\mu)$ eine $\sigma$-Algebra und $\mu$ ist ein vollständiges Maß auf $\script{M}(\mu)$.
\end{frame}


\begin{frame}
\frametitle{Lemma II.8}
$\mu$ äußeres Maß, $A_i \in \script{M}(\mu), i \in \mathbb{N}$.\\
      Dann gelten:
      \begin{enumerate}[label=\roman*)]
        \item Aus $A_1 \subseteq ... \subseteq A_i \subseteq A_{i+1} \subseteq ...$ folgt $\mu(\bigcup\limits_{i \in \mathbb{N}} A_i) = \lim\limits_{i \to \infty} \mu(A_i)$
        \item Aus $A_1 \supseteq ... \supseteq A_i \supseteq A_{i+1} \supseteq ...$ mit $\mu(A_1) < \infty$ folgt $\mu(\bigcap\limits_{i \in \mathbb{N}} A_i) = \lim\limits_{i \to \infty} \mu(A_i)$
      \end{enumerate} 
\end{frame}


\begin{frame}
\frametitle{Def. II.9 / X-stabil}
Ein Mengensystem $\script{A} \subseteq \script{P}(X)$ heißt $\bm{\bigcup}$\textbf{-stabil} (bzw. $\bm{\bigcap}$\textbf{-stabil}, $\bm{\setminus}$\textbf{-stabil}), wenn $A \cup B \in \script{A}$ (bzw. $A \cap B \in \script{A}$, $A \setminus B \in \script{A}$) $\forall A,B \in \script{A}$ gilt.\\
\medskip
$\bigcup$-stabil impliziert Stabilität bzgl. endlicher Vereinigung. Ebenso $\bigcap$-stabil.
\end{frame}


\begin{frame}
\frametitle{Def. II.10 / Ring / Algebra}
Ein Mengensystem $\script{R}\subset\script{P}(X)$ heißt \textbf{Ring} über $X$, falls:
      \begin{enumerate}[label=\roman*)]
        \item $\emptyset \in \script{R}$
        \item $A,B \in \script{R} \implies A \setminus B \in \script{R}$
        \item $A,B \in \script{R} \implies A \cup B \in \script{R}$
      \end{enumerate}
      
      $\script{R}$ heißt \textbf{Algebra}, falls zusätzlich $X \in \script{R}$.\\
      \medskip
Für $A,B \in \script{R}$ gilt: $A \cap B = A \setminus (A \setminus B) \in \script{R}$\\
      Ringe sind $\bigcup$-stabil, $\bigcap$-stabil, $\setminus$-stabil
\end{frame}


\begin{frame}
\frametitle{Def. II.11 (Im Aufschrieb II.10) / Prämaß}
Sei $\script{R} \subseteq \script{P}(X)$ Ring. Eine Funktion $\lambda: \script{R} \to [0, \infty]$ heißt \textbf{Prämaß} auf $\script{R}$, falls:
      \begin{enumerate}[label=\roman*)]
        \item $\lambda(\emptyset) = 0$
        \item Für $A_i \in \script{R}, i \in \mathbb{N}$, paarweiße disjunkt mit $\bigcup\limits_{i \in \mathbb{N}} A_i \in \script{R}$ gilt:\\
        $\lambda(\bigcup\limits_{i \in \mathbb{N}} A_i) = \sum\limits_{i \in \mathbb{N}} \lambda(A_i) $
      \end{enumerate}
      \medskip
$\sigma$-subadditiv, subadditiv, $\sigma$-endlich, endlich, monoton, Nullmenge und fast-überall werden wie für Maße definiert.
\end{frame}


\begin{frame}
\frametitle{Def. II.12 (Im Aufschrieb II.11) / Fortsetzung}
$\lambda$ Prämaß auf Ring $\script{R} \subseteq \script{P}(X)$. Ein äußeres Maß $\mu$ auf $X$ (bzw. ein Maß auf $\script{A}$) heißt \textbf{Fortsetzung} von $\lambda$, falls gilt:
      \begin{enumerate}[label=\roman*)]
        \item $\mu|_\script{R} = \lambda$, d.h. $\mu(A) = \lambda(A) \ \forall A \in \script{R}$
        \item $\script{R} \subseteq \script{M}(\mu)$ (bzw. $\script{R} \subset \script{A}$), d.h. alle $A \in \script{R}$ sind $\mu$-messbar
      \end{enumerate}
\end{frame}


\begin{frame}
\frametitle{induziertes äußeres Maß / Caratheodory-Fortsetzung}

\end{frame}


\begin{frame}
\frametitle{Lemma II.14 (Im Aufschrieb II.13)}
$\lambda: \script{R} \to [0, \infty]$ Prämaß auf Ring $\script{R} \subseteq \script{P}(X)$. Sei $\mu: \script{P}(X) \to [0, \infty]$ das in Satz II.3 aus $\script{R}$ konstruierte äußere Maß, d.h. $\forall E \subseteq X:$
      \begin{align*}
        \mu(E) := inf\{\sum\limits_{i \in \mathbb{N}} \lambda(A_i) \ | \ A_i \in \script{R}, E \subseteq \bigcup\limits_{i \in \mathbb{N}} A_i\}
      \end{align*}
      Dann ist $\mu$ eine Fortsetzung von $\lambda$.\\
      $\mu$ heißt \textbf{induziertes äußeres Maß} oder \textbf{Caratheodory-Fortsetzung} von $\lambda$.
\end{frame}

% Vorlesung 6

\begin{frame}
\frametitle{}

\end{frame}


\begin{frame}
\frametitle{}

\end{frame}




\end{document}
